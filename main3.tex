\documentclass{article}
\usepackage[utf8]{inputenc}
\usepackage[left=0.85in,top=1in,right=0.85in,bottom=1in]{geometry} 

\title{Learning How to Play Blackjack DeCal Syllabus}
\author{Tina Tseng, Austin Tseng, Sean Kim}
\date{Spring Semester 2019}

\begin{document}

\maketitle

\section{General Information}

Course: Learning How to Play Blackjack DeCal Syllabus
\newline
1 Unit Course 
\newline
Student Facilitators: Tina Tseng, Austin Tseng, Sean Kim
\newline
Lecture: Tuesdays 6:00-7:00pm
\newline
Location: Dwinelle 206
\newline
Grading: P/NP


\section{Course Overview}
  This course will teach students interested in Blackjack how to play as well as teach strategies on the best decisions to make in common game situations. By the end of the DeCal, students will be able to play at an intermediate-advanced level. The class time will be split up with half the time as instruction and the other half with hands-on play with what was taught in class.
   
  
\section{The Rule}
Blackjack is a popular casino game that is played between the player and the dealer. The objective is to try and beat the dealer by picking up a score of 21 on the first two cards, which is why the game is also referred to as 21. You can do this by:

\begin
Scoring 21 on the first two cards dealt, as long as the dealer does not have the same hand. This hand is called a blackjack.
Beating the dealer’s final score without getting over 21.
Allowing the dealer to extend his hand with additional cards and getting his score to go over 21.

\section{Assignments}
    \item Reading Reflection: These are two pages in length reading reflections based on the movie: 21. A reading reflection includes your interpretations, impressions, observations, and opinions (they are not summaries).

    \item Final Reflection: One page in length. Describes how you have learned in this course and/or advice for this course.  
    
 \section{Schedule}
\begin{tabular}{ |p{2cm}| p{9.5cm}| p{5cm}|}
\hline
\bf Date & \bf Topic & \bf Description\\
\hline
2/5 & \bf Lecture1: Introduction to Class, Overview of Syllabus & Attendance\\
\hline
2/12 &\bf Lecture2: What are the card values in BLACKJACK & Attendance\\
\hline 
2/19 & \bf Lecture3: How to Play Blackjack : Basic Rules & Attendance\\
\hline 
2/26 & \bf Lecture4: Basic Strategy and Player’s Moves in Blackjack & Attendance\\
\hline 
3/5  & \bf Lecture5: Basic Blackjack Strategy & Attendance\\
\hline 
3/12  & \bf Lecture6: Blackjack Hand Signals & Attendance\\
\hline
3/19  & \bf Lecture7: What are the different types of moves in blackjack games? & Attendance\\
\hline
3/26  & \bf Lecture8: Spring Recess & NO CLASS \\
\hline
4/2  & \bf Lecture9: Evaluating Games & Attendance\\
\hline
4/9  & \bf Lecture10: Movie:21 & Attendance\\
\hline
4/16  & \bf Lecture11: Movie:21  & Attendance\\
\hline
4/23  & \bf Lecture12: The Game Master & Reading Reflection; Attendance\\
\hline
4/30  & \bf Lecture13: Blackjack Tournament & Final Reflection; Attendance\\
\hline

\end{tabular}

\newline  
\section{Grading Policy}
Grading for this class is P/NP and will be based on attendance and participation. Here are the details for how attendance and participation will be graded:
\newline
\begin{itemize}
	\item Attendance (30 points): Each student will be allowed three absences throughout the semester. However, more absences than that will result in a NP. Students must attend the final day of class where a tournament-style rounds of Blackjack will be played.
    \item Participation (30 points): All students are expected to participate in class. Although students are not expected to have prior knowledge of playing, effort should be given from students towards learning. 
    \item Reading Reflection (20 points)
    \item Final Reflection (20 points)
    
\end{itemize}
There are 100 points total. Students who received 70 points or above in class will receive a P.  
\end{document}
