\documentclass{article}
\usepackage[utf8]{inputenc}
\usepackage[left=0.85in,top=1in,right=0.85in,bottom=1in]{geometry} 

\title{Learning How to Play Blackjack DeCal Syllabus}
\author{Tina Tseng, Austin Tseng, Sean Kim}
\date{Spring Semester 2019}

\begin{document}

\maketitle

\section{General Information}

Course: Learning How to Play Blackjack DeCal Syllabus
\newline
1 Unit Course 
\newline
Student Facilitators: Tina Tseng, Austin Tseng, Sean Kim
\newline
Lecture: Tuesdays 6:00-7:00pm
\newline
Location: Dwinelle 206
\newline
Grading: P/NP


\section{Course Overview}
  This course will teach students interested in Blackjack how to play as well as teach strategies on the best decisions to make in common game situations. By the end of the DeCal, students will be able to play at an intermediate-advanced level. The class time will be split up with half the time as instruction and the other half with hands-on play with what was taught in class.
   
  
\section{The Rule}
Blackjack is a popular casino game that is played between the player and the dealer. The objective is to try and beat the dealer by picking up a score of 21 on the first two cards, which is why the game is also referred to as 21. You can do this by:

\begin
Scoring 21 on the first two cards dealt, as long as the dealer does not have the same hand. This hand is called a blackjack.
Beating the dealer’s final score without getting over 21.
Allowing the dealer to extend his hand with additional cards and getting his score to go over 21.

\section{Schedule}
\begin{tabular}{ |p{2cm}| p{9.5cm}| p{4.5cm}|}
\hline
\bf Date & \bf Topic & \bf Description\\
\hline
2/5 & \bf Lecture1: Introduction to Class, Overview of Syllabus & \\
\hline
2/12 &\bf Lecture2: &\\
\hline 
2/19 & \bf Lecture3: &\\
\hline 
2/26 & \bf Lecture4: &\\
\hline 
3/5  & \bf Lecture5: &\\
\hline 
3/12  & \bf Lecture6: &\\
\hline
3/19  & \bf Lecture7: &\\
\hline
3/26  & \bf Lecture8: Spring Recess & NO CLASS \\
\hline
4/2  & \bf Lecture9: &\\
\hline
4/9  & \bf Lecture10: &\\
\hline
4/16  & \bf Lecture11: &\\
\hline
4/23  & \bf Lecture12: &\\
\hline
4/30  & \bf Lecture13: &\\
\hline

\end{document}
